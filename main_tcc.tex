% +------- CÓDIGO FONTE TCC -------+
% 	Autor: Arllem Farias, 2017.
% 	E-mail: arllemfarias@ufam.edu.br
%   Revisor: João Victor Lima Lopes
%   E-mail: jvllopess@gmail.com
% +--------------------------------+

% Ítens como: Ficha Catalográfica, Folha de Aprovação Assinada e Lombada, devem ser adicionadas posteriormente
% e descomentar as linhas que as adicionam (Linhas 132,134 e 135).

% Classe do documento --> \documentclass[OPÇÕES]{CLASSE}
\documentclass[
			   a4paper, % Tipo de papel 
			   oneside, % Impressão em apenas um lado da folha
			   12pt     % Tamanho da fonte
			   ]{book}  % Classe do documento

% Pacotes --> \usepackage[OPÇÕES]{PACOTE}
\usepackage[utf8]{inputenc} % Codificação do documento (conversão automática dos acentos) 
\usepackage[brazil]{babel}  % Traduz palavras chaves para o PT-BR (ex.: abstract->resumo)
\usepackage{indentfirst}    % Indenta o primeiro parágrafo de cada seção
\usepackage{setspace}		% Possibilita a alteração do espaçamento entre linhas
\usepackage{graphicx}       % Possibilita a inserção de figuras
\usepackage{subcaption}     % Possibilita a inserção de subfiguras
\usepackage{pdfpages}       % Possibilita a inserção de páginas em pdf
\usepackage{amsmath}        % Inclui funções adicionais no ambiente matemático \eqref{•} \dfrac{•}{•}
\usepackage{amssymb} 		% Símbolos adicionais no documento \mathbb{R}
\usepackage{mathrsfs}       % Símbolos da transformadas de Laplace e Fourier \mathscr{•}
\usepackage{float}			% Possibila posicionar tabelas e figuras em uma posição específica [H]
\usepackage[numbers]{natbib}% Inclui mais possibilidades de citações \citep{•}
\usepackage{fancyhdr}       % Possibilita a alteração de cabeçalho e rodapé
\usepackage{longtable}      % Possiblita a quebra de tableas em duas páginas
\usepackage{multirow}		% Possibilita multiplas linhas em tabelas
\usepackage{array}			% Possibilita o uso do comando \newcolumntype{•}[•]
\usepackage{pdflscape}      % Possibilita a inserção de páginas em modo paisagem
\usepackage{listings}		% Possibilita inserir códigos fontes (C++, Java, ...)
\usepackage{slashbox}       % Adiciona o comando \backslashbox{•}{•} usado em tabelas
\usepackage{arydshln}       % Possibilita inserir linhas pontilhadas em tabelas 

%\usepackage[inline]{showlabels} % Mostra os labels das equações
%\usepackage[notcite,notref]{showkeys} % Mostra todo os labels
%\usepackage{lipsum} % preenchimento automático de textos

% Modifica os itens do sumário
\usepackage[nottoc,
			notlof,
			notlot]{tocbibind}

% Configura as margens das páginas
\usepackage[top    = 3cm,
			bottom = 2cm,
			left   = 3cm,
			right  = 2cm]{geometry}

% Possibilita hiperlinks no texto
\usepackage[pdftex,
			%backref,
			linktocpage = false,
			colorlinks  = true,
			linkcolor   = blue,
			anchorcolor = blue,
			citecolor   = blue,
			urlcolor    = blue]{hyperref}

% Comandos auxiliares --> \nomecomando{COMANDO}{•}
\newcolumntype{C}[1]{>{\centering\let\newline\\\arraybackslash\hspace{0pt}}m{#1}} % Tabelas: {|C{2cm}|C{5cm}|}
\newcolumntype{L}[1]{>{\let\newline\\\arraybackslash\hspace{0pt}}m{#1}} % Tabelas: {|L{2cm}|L{5cm}|}
\newcommand*{\doi}[1]{DOI: \href{http://dx.doi.org/#1}{#1}} % Usado nas referencias

\setcounter{secnumdepth}{3} % Inclui a numeração de \subsubsection{•} no documento
\setcounter{tocdepth}{3}    % Inclui a \subsubsection{•} no sumário
\setstretch{1.5}			% Configura o espaçamento entre linhas \usepackage{setspace}

\pagestyle{fancy} 			% Configura a página para incluir o cabeçalho e rodapé
\lhead{{\footnotesize\leftmark}} % Cabeçalho esquerdo
\chead{}				         % Cabeçalho central
\rhead{\thepage}				 % Cabeçalho direito
\fancyfoot{}					 % Rodapé vazio

% Definição de novas cores
\definecolor{mygreen}{RGB}{0, 115, 0}
\definecolor{mylilas}{RGB}{170,55,240}

\lstset{ % -> \usepackage{listings}
  language=Matlab,
  basicstyle=\ttfamily\small, 
  morekeywords={matlab2tikz},
  keywordstyle=\color{blue}, 
  stringstyle=\color{mylilas}, 
  commentstyle=\color{mygreen}, 
  extendedchars=true,
  showspaces=false,
  showstringspaces=false,
  numbers=left,
  numberstyle=\tiny,
  breaklines=true,
  breakautoindent=true,
  captionpos=b,
  xrightmargin=0pt,
  xleftmargin=15pt,
}


% Novos comandos --> \newcommand{COMANDO}{DEFINIÇÃO}
\newcommand{\instituicao}{
				UNIVERSIDADE FEDERAL DO AMAZONAS\\
				FACULDADE DE TECNOLOGIA\\
				ENGENHARIA DA COMPUTAÇÃO}
\newcommand{\titulo}{
				TÍTULO AQUI}
\newcommand{\apresentacao}{
				Monografia apresentada à Coordenação do Curso
				de Engenharia da Computação da Universidade Federal
				do Amazonas, como parte dos requisitos necessários
				à obtenção do título de Engenheiro de Computação.}
\newcommand{\autor}{
				AUTOR AQUI}
\newcommand{\local}{
				MANAUS-AM\\ANO}
\newcommand{\orientador}{
				ORIENTADOR AQUI}

%-----------------------------------------------------
%\usepackage{ulem}
%\newcommand{\commentib}[1]{{\color{red} [IB: #1]}}
%\newcommand{\corrigir}[1]{{\color{violet}\uwave{#1}}}
%-----------------------------------------------------

\begin{document}

\begin{titlepage}

\begin{center}

\begin{figure}[t]
	\centering
	\includegraphics[scale=0.7]{pasta1_figuras/logo_ufam.eps}
\end{figure}

\textbf {\instituicao}

\vfill
\textbf{\titulo}

\vfill
\textbf{\autor}

\vfill
\textbf{\local}

\end{center}

\end{titlepage}

%\begin{titlepage}
	
\includepdf{parte1_pre-textuais/lombada_tcc.pdf}
		
\end{titlepage}



\thispagestyle{empty}

\begin{center}

\autor

\vfill {\Large \titulo}

\vfill{
\begin{flushright}
	\begin{minipage}{8cm} 
		\apresentacao
	\end{minipage}
\end{flushright}
}

\vfill Orientador: \orientador

\vfill	\local

\end{center}

%\thispagestyle{empty}

\includepdf{parte1_pre-textuais/ficha_catalografica_ufam.pdf}

%\thispagestyle{empty}

\includepdf{parte1_pre-textuais/folha_de_aprovacao_assinada.pdf}

\thispagestyle{empty}

\vspace*{\fill}
\begin{flushright}
	\begin{minipage}{8cm}
		\textit{
			\qquad DEDICATÓRIA AQUI
		}
	\end{minipage}
\end{flushright}

\chapter*{Agradecimentos}
\thispagestyle{empty}

AGRADECIMENTOS AQUI.
\thispagestyle{empty}

\vspace*{\fill}
\begin{flushright}
	\begin{minipage}{8cm}
		\textit{
			EPÍGRAFE AQUI
			\\\\
			\rightline{(AUTOR AQUI, TÍTULO DA OBRA.)}
		}
	\end{minipage}
\end{flushright}



\chapter*{Resumo}
\thispagestyle{empty}

RESUMO AQUI

\vspace{50pt}

\paragraph{Palavras-chave:} PALAVRAS CHAVES AQUI.

\chapter*{Abstract}
\thispagestyle{empty}

ABSTRACT AQUI

\vspace{50pt}

\paragraph{Keywords:} KEYWORDS HERE.



\pagenumbering{roman}

\listoffigures

%\thispagestyle{empty}

\include{parte1_pre-textuais/item12_lista_de_tabelas}
\markboth{\MakeUppercase{Lista de Abreviaturas e Siglas}}{\MakeUppercase{Lista de Abreviaturas e Siglas}}

\chapter*{Lista de Abreviaturas e Siglas}

\begin{longtable}{lL{14cm}L{\textwidth}}

\textbf{SIGLA}  & NOME EXPANDIDO -- do inglês \textit{\textbf{S}I \textbf{G}l \textbf{A}} & \\


\end{longtable}

\markboth{\MakeUppercase{Lista de Símbolos}}{\MakeUppercase{Lista de Símbolos}}

\chapter*{Lista de Símbolos}

\begin{spacing}{1.45}

\noindent \textbf{Símbolos Matemáticos}

\begin{longtable}{L{1.5cm}L{14cm}L{\textwidth}}

$\mathbb{R}$ & conjunto dos números reais & \\



\end{longtable}

\end{spacing}

\include{parte1_pre-textuais/item15_sumario}

\pagenumbering{arabic}

\chapter{Introdução} %Contextualização, motivação e justificativa.

INTRODUÇÃO AQUI

% Fim Capítulo
\chapter{Fundamentação Teórica} \label{cap2}
FUNDAMENTAÇÃO TEÓRICA AQUI

% Fim Capítulo
\chapter{Título do Capítulo Aqui} \label{cap3}
MODELAGEM AQUI
% Fim Capítulo

\chapter{Título do Capítulo Aqui} \label{cap4}
METODOLOGIA AQUI

% Fim Capítulo
\chapter{Título do Capítulo Aqui} \label{cap5}
RESULTADOS AQUI
% Fim Capítulo

\chapter{Conclusão} \label{cap6}

CONCLUSÃO AQUI

% Fim Capítulo


% FOI RENOMEADO O ARQUIVO: main_tcc.bbl para main_tcc.bbl_OLD
%\bibliographystyle{IEEEtranN} % Ordem de citação
\bibliographystyle{humannat} % Ordem alfabética
%\bibliographystyle{dinat}    % Ordem alfabética
%\bibliographystyle{plainnat} % Ordem alfabética
%\bibliographystyle{apa}      % Ordem alfabética

\bibliography{referencias}

\appendix

%\chapter{TITULO APÊNDICE} \label{apendice:a}




\end{document}
